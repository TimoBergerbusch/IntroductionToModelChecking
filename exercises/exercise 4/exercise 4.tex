\documentclass[11pt]{article}
\usepackage{amsmath} % Math
\usepackage{amssymb} % Math symbols
\usepackage[english]{babel} % Language
\usepackage{fancyhdr} % Header
\usepackage[a4paper, total={15cm, 20cm}]{geometry} % Dimensions of the paper and the text area
\usepackage[utf8]{inputenc} % encoding in UTF, needed for umlauts if German
\usepackage{mathtools} % Text above arrows
\usepackage{msc} % Drawing MSCs
\usepackage{multicol} % Multiple columns
\usepackage[explicit]{titlesec} % Automatic section titles
\usepackage{tikz} % Diagrams
\usetikzlibrary{arrows.meta, automata, shapes, matrix,positioning}
\usepackage{enumitem}   % Enumeration item
\usepackage{float}
\usepackage{verbatim}


% Other packages that might be useful in the future
%\usepackage{lingmacros}
%\usepackage{tree-dvips}
%\usepackage{ulem}
%\usepackage{amsthm}
%\usepackage{amsbsy}
%\usepackage{textcomp,gensymb}
%\usepackage{graphicx}
%\usepackage{mathtools}

% Custom variant of msc environment:
% - No "msc" keyword, longer partial messages
% - Increased vertical distance between messages
% - Less distance to the frame left and right
% - Less distance between header and processes
% - Less distance between footer and frame
% - Passing all given options down to the msc environment
\newenvironment{cmsc}[1][]{\msc[msc keyword={}, self message width=1.1cm, level height=0.6cm, environment distance=1.2cm, head top distance=0.75cm, foot distance=0.5cm, #1]}{\endmsc}

% No indentation at new paragraphs
\setlength{\parindent}{0pt}

% Distance between columns
\setlength{\columnsep}{1cm}
% Vertical line between columns
\setlength{\columnseprule}{0.5pt}
\def\columnseprulecolor{\color{gray}}

% Settings
\newcommand{\sheetNr}{4}
\newcommand{\red}[1]{\color{red}#1\color{black}}

%% Header
\fancyhf{}
\pagestyle{fancy}
\lhead{Model Checking Exercise Sheet \sheetNr}
\rhead{Aaron Grabowy: 345766\\ Timo Bergerbusch: 344408\\ Felix Linhart: 318801}
\setlength{\headheight}{40pt}

%% Automatic section titles
\titleformat{\section}{\normalfont\Large\bfseries}{}{0em}{Exercise #1}
\titleformat{\subsection}{\normalfont\large\bfseries}{}{0em}{#1)}

\begin{document}
	
\section{1}
\subsection{a}

\section{2}
\subsection{a}
The statement does not hold.\\
Counterexample: 

\begin{tikzpicture}[shorten >=1pt,node distance=2cm,on grid,auto] 
	\node[] (LabelTS1) {$TS_1$:};
	\node[state, initial, initial text = , below = of LabelTS1, label = below:$\emptyset$] (s0) {$s_0$};
	\node[state, label = below:$\emptyset$, right = of s0] (s1) {$s_1$};
	
	\node[right = 5cm of LabelTS1] (LabelTS2) {$TS_2$:};
	\node[state, initial, initial text = , below = of LabelTS2, label = below:$\emptyset$] (t0) {$t_0$};
	\node[state,label = below:$\emptyset$, right = of t0] (t1) {$t_1$};

	\node[right = 5cm of LabelTS2] (LabelTS12) {$TS_1 || TS_2$:};	
	\node[state, initial, initial text = , below = of LabelTS12, label = below:$\emptyset$] (u0) {$\langle s_0, t_0 \rangle$};
	\node[state,label = below:$\emptyset$, right = of u0] (u1) {$\langle s_1, t_1 \rangle$};
		
	\path[->] (s0) edge[bend left = 20] node {$\tau$} (s1);
	\path[->] (s1) edge[bend left = 20] node {$\tau$} (s0);
	\path[->] (t0) edge node {$\tau$} (t1);
	\path[->] (u0) edge node {$\tau$} (u1);
\end{tikzpicture}\\
Then: \begin{itemize}
	\item $trace(TS_1) = \{\emptyset^\omega\}$
	\item $trace(TS_1) = \{\emptyset\emptyset\}$
	\item $trace(TS_1 || TS_2) = \{\emptyset\emptyset\}$
	\item $\Rightarrow \emptyset^\omega \in Traces(TS_1)$, but $\emptyset^\omega \notin Traces(TS_1||TS_2)$
\end{itemize}

\subsection{b}
The statement does not hold.\\
Counterexample: 

\begin{tikzpicture}[shorten >=1pt,node distance=2cm,on grid,auto] 
\node[] (LabelTS1) {$TS_1$:};
\node[state, initial, initial text = , below = of LabelTS1, label = below:$\emptyset$] (s0) {$s_0$};
\node[state, label = below:$\{a\}$, right = of s0] (s1) {$s_1$};

\node[right = 5cm of LabelTS1] (LabelTS2) {$TS_2$:};
\node[state, initial, initial text = , below = of LabelTS2, label = below:$\emptyset$] (t0) {$t_0$};
\node[state,label = below:$\emptyset$, right = of t0] (t1) {$t_1$};

\node[right = 5cm of LabelTS2] (LabelTS12) {$TS_1 || TS_2$:};	
\node[state, initial, initial text = , below = of LabelTS12, label = below:$\emptyset$] (u0) {$\langle s_0, t_0 \rangle$};
\node[state,label = below:$\{a\}$, right = of u0] (u1) {$\langle s_1, t_1 \rangle$};

\path[->] (s0) edge[bend left = 20] node {$\tau$} (s1);
\path[->] (s1) edge[bend left = 20] node {$\alpha$} (s0);
\path[->] (t0) edge[loop above] node {$\alpha$} (t0);
\path[->] (t0) edge node {$\tau$} (t1);
\path[->] (u0) edge node {$\tau$} (u1);
\end{tikzpicture}\\
Then: \begin{itemize}
	\item $trace(TS_1) = \{(\emptyset\{a\})^\omega\}$
	\item $trace(TS_1) = \{\emptyset\emptyset\}$
	\item $trace(TS_1 || TS_2) = \{\emptyset\{a\}\}$
	\item $\Rightarrow \emptyset\{a\} \in Traces(TS_1|| TS_2)$, but $\emptyset\{a\} \notin Traces(TS_1)$
\end{itemize}

\subsection{c}

The statement holds.\\
Let $Traces(TS_1) \subseteq Traces(TS_2)$. Then $Traces(TS_1) \setminus Traces(TS_2) = \emptyset$.\\
Further let $t$ be an arbitrary trace, with $t \in Traces(TS_1) \cap Traces(TS_2)$:\\
If $t \text{ is a } \mathcal{F}\text{-fair execution}$ then$t\in FairTraces_\mathcal{F}(TS_1)$ and $t\in FairTraces_\mathcal{F}(TS_2)$\\
If $t \text{ is not a } \mathcal{F}\text{-fair execution}$ then $t\notin FairTraces_\mathcal{F}(TS_1)$ and $t\notin FairTraces_\mathcal{F}(TS_2)$\\
So $FairTraces(TS_1) \subseteq FairTraces(TS_2)$.
\subsection{d}

Let $Traces(TS_1)\subseteq Traces(TS_2)$,$E = $eventually $a$ will occur, with $a \in Act_2$, but $a \notin Act_1$, and $TS_2 \models_F E$.\\

Through the lecture we know, that liveness properties are not effected by fairness assumptions.\\
So $TS \models_\mathcal{F} E \Leftrightarrow TS \models E$ for any liveness property $E$.\\

So $TS_1 \models_\mathcal{F} E \Leftrightarrow TS_1 \models E$. Let a finite word $\sigma$ be given, not containing an $a$. 
Since $a\notin Act_1$ we cannot find a $\sigma'$ s.t. $\sigma\sigma' \in E$. So $TS_1 \not\models E$ and therefore $TS_1 \not\models_\mathcal{F} E$.

%Further if we can find for every $\sigma \in 2^{AP}$ a $\sigma' \in (2^{AP})^\omega$, s.t. $\sigma\sigma' \in E$ then we also can find such $\simga'$ for 

\section{3}
\subsection{a}
\begin{itemize}
	\item[$E_a$:] $TS \not\models E_a$, since $\sigma = (\emptyset \{b\})^\omega \in Traces(TS)$, but $\sigma$ obviously contains no $a$.
	\item[$E_b$:] $TS \not\models E_b$, since $\sigma' = (\emptyset \{a\})^\omega \in Traces(TS)$, but $\sigma'$ obviously contains no $b$.
	\item[$E'$:] $TS \not\models E'$, since $\sigma'' = (\emptyset\{a\}\{a,b\}\emptyset)^\omega \in Traces(TS)$, but there ex. a $i=1$, such that $A_i=\{a\}, A_{i+1}=\{a,b\}$ and $A_{i+2}=\emptyset$.
\end{itemize}

\subsection{b}
% BEGIN: E_a
\subsubsection{$E_a$}\label{here}
\begin{itemize}
%	\setlength{\itemindent}{1cm}
	\item[$B=B_1$:] $TS \models_{\mathcal{F}_{strong}^1} E_a$\\
		There is no path from an initial state s.t. never an $\alpha$-transition is enabled.
		Therefore we have to take an $\alpha$ infinitely many times. So we either go the loop $s_1s_2s_4$ or the loop $s_1s_3s_4$ both fulfilling the premise and conclusion of strong fairness. But then we infinitely often visit state $s_4$ containing an $a$ in its label. So we have have $a$ infinitely often.
	\item[$B=B_2$:] $TS \not\models{\mathcal{F}_{strong}^2} E_a$\\
	One path that fulfils the strong fairness condition is $\pi = s_1 s_3 s_1 s_3\dots$.\\
	But $trace(\pi) = (\emptyset\{b\})^\omega$ does not have infinitely many $a$'s.
	\item[$B=B_3$:] $TS \not\models{\mathcal{F}_{strong}^3} E_a$\\
	One path that fulfils the strong fairness condition is $\pi = s_1 s_3 s_1 s_3\dots$.\\
	But $trace(\pi) = (\emptyset\{b\})^\omega$ does not have infinitely many $a$'s.
\end{itemize}

% BEGIN: E_b
\subsubsection{$E_b$}
\begin{itemize}
	%	\setlength{\itemindent}{1cm}
	\item[$B=B_1$:] $TS \not\models{\mathcal{F}_{strong}^1} E_b$\\
	One path that fulfils the strong fairness condition is $\pi = s_1s_2s_1s_2\dots$\\
	But $trace(\pi) = (\emptyset\{a\})^\omega$ does not have infinitely many $b$'s.
	\item[$B=B_2$:] $TS \not\models{\mathcal{F}_{strong}^2} E_b$\\
	One path that fulfils the strong fairness condition is $\pi = s_1s_2s_1s_2\dots$\\
	But $trace(\pi) = (\emptyset\{a\})^\omega$ does not have infinitely many $b$'s.
	\item[$B=B_3$:] $TS \models{\mathcal{F}_{strong}^3} E_b$\\
	There is just one path fulfilling the strong fairness condition: $\pi= s_1 s_3 s_4 s_1 s_3 s_4 \dots$\\
	Then $trace(\pi) = (\emptyset\{b\}\{a,b\})^\omega$ has infinitely many $b$'s.
\end{itemize}

% BEGIN: E'
\subsubsection{$E'$}
\begin{itemize}
	%	\setlength{\itemindent}{1cm}
	\item[$B=B_1$:] $TS \not\models{\mathcal{F}_{strong}^1} E'$\\
	One path that fulfils the strong fairness condition is $\pi = s_1s_2s_4s_1s_2s_4\dots$\\
	But $trace(\pi) = (\emptyset\{a\}\{a,b\})^\omega \not\models E'$.
	\item[$B=B_2$:] $TS \not\models{\mathcal{F}_{strong}^2} E'$\\
	One path that fulfils the strong fairness condition is $\pi = s_1s_2s_4s_1s_2s_4\dots$\\
	But $trace(\pi) = (\emptyset\{a\}\{a,b\})^\omega \not\models E'$.
	\item[$B=B_3$:] $TS \not\models{\mathcal{F}_{strong}^2} E'$\\
	One path that fulfils the strong fairness condition is $\pi= s_1s_3s_4 s_1s_2s_4s_1s_3s_4 s_1s_2s_4 \dots$\\
	But $traces(\pi)=(\emptyset\{b\}\{a,b\}\emptyset\{a\}\{a,b\})^\omega \not\models E'$
\end{itemize}

\subsection{c}

% BEGIN: E_a
\subsubsection{$E_a$}
\begin{itemize}
	%	\setlength{\itemindent}{1cm}
	\item[$B=B_1$:] $TS \models_{\mathcal{F}_{weak}^1} E_a$\\
		The only path not containing infinitely many $a$'s is the path $\pi= s_1s_3s_1s_3\dots$. Then $\pi$ fulfils the conditions of weak fairness, but not the conclusion and therefore is so eligible path. So we have to take $s_3 \xrightarrow{\alpha}a_4$ or $s_1 \xrightarrow{\alpha} s_2$ infinitely often leading to a state having $a$ in the label. So we have infinitely many $a$'s.
	\item[$B=B_2$:] $TS \not\models{\mathcal{F}_{weak}^2} E_a$\\
		One path that fulfils the weak fairness condition is $\pi = s_1 s_3 s_1 s_3\dots$. We fulfil the premise, but by taking $\beta$ every second time we also fulfil the conclusion.\\
		But $trace(\pi) = (\emptyset\{b\})^\omega$ does not have infinitely many $a$'s.
	\item[$B=B_3$:] $TS \not\models{\mathcal{F}_{weak}^3} E_a$\\
		There is no path in $TS$, such that the premise holds. So all paths are fair with respect to the stated fairness condition. But again we can take the path $\pi = s_1 s_3 s_1 s_3\dots$ and observe, that $trace(\sigma) =(\emptyset\{b\})^\omega$ does not contain infinitely many $a$'s.
\end{itemize}

% BEGIN: E_b
\subsubsection{$E_b$}
\begin{itemize}
	%	\setlength{\itemindent}{1cm}
	\item[$B=B_1$:] $TS \not\models{\mathcal{F}_{weak}^1} E_b$\\
		One path that does not fulfil the weak fairness premise is $\pi = s_1s_2s_1s_2\dots$. Since it does not satisfy the premise is satisfies the whole condition.\\
		But $trace(\pi) = (\emptyset\{a\})^\omega$ does not have infinitely many $b$'s.
	\item[$B=B_2$:] $TS \not\models{\mathcal{F}_{weak}^2} E_b$\\
		One path that fulfils the weak fairness premise and conclusion is $\pi = s_1s_2s_1s_2\dots$\\
		But $trace(\pi) = (\emptyset\{a\})^\omega$ does not have infinitely many $b$'s.
	\item[$B=B_3$:] $TS \not\models{\mathcal{F}_{weak}^3} E_b$\\
		There is not path s.t. contentiously $\beta$ transitions are enabled. So the premise does not hold and therefore the weak fairness is fulfilled. 
		But obviously there are paths not fulfilling the property, like $\pi = s_1s_2s_1s_2\dots$, with $trace(\pi) = (\emptyset\{a\})^\omega$ does not have infinitely many $b$'s.
\end{itemize}

% BEGIN: E'
\subsubsection{$E'$}
\begin{itemize}
	%	\setlength{\itemindent}{1cm}
	\item[$B=B_1$:] $TS \not\models{\mathcal{F}_{strong}^1} E'$\\
		One path that does not fulfil the weak fairness premise is $\pi=s_1s_2s_4s_1s_2s_4\dots$. Since it does not satisfy the premise is satisfies the whole condition.\\
		But $trace(\pi) = (\emptyset\{a\}\{a,b\})^\omega$ does not satisfy the property.
	\item[$B=B_2$:] $TS \not\models{\mathcal{F}_{strong}^2} E'$\\
		One path that does not fulfil the weak fairness premise is $\pi=s_1s_2s_4s_1s_2s_4\dots$. Since it does not satisfy the premise is satisfies the whole condition.\\
		But $trace(\pi) = (\emptyset\{a\}\{a,b\})^\omega$ does not satisfy the property.
	\item[$B=B_3$:] $TS \not\models{\mathcal{F}_{strong}^2} E'$\\
		There is not path s.t. contentiously $\beta$ transitions are enabled. So the premise does not hold and therefore the weak fairness is fulfilled.\\
		But for the path $\pi=s_1s_2s_4s_1s_2s_4\dots$ the trace looks like the following $trace(\pi)= (\emptyset\{a\}\{a,b\})^\omega$ and obviously does not satisfy the property.
\end{itemize}

\end{document}
